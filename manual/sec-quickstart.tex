\section{Getting Started}

\modeltest provides graphical and a command console interfaces.

\subsection{Compilation}

Sources can be compiled for every major Operating System, including Linux, Windows, and Mac OS X. For convenience, with each release you will find binaries for each of these systems.
Nonetheless, it might happen that for certain distributions only some of theme are available, for example if the realease fixes a bug affecting one single OS.

This tool is distributed under GPL v3 license. 
The source code is freely available at github repository: \url{https://github.com/ddarriba/modeltest}.

If you clone the repository, make sure to clone also libpll and pll-modules submodules with `--recursive` flag, or call `git submodule init --recursive` afterwards.

For most users, download the latest release and call the build script, {\em build.sh}. 
It should work correctly and (by default) compile all dependencies and place the final binaries under `build` directory.

\begin{itemize}
\item Run {\em build.sh} with no arguments for compiling all dependencies and modeltest in 3 flavors: {\em modeltest-ng} (command console execution), {\em modeltest-mpi} (MPI version), and {\em modeltest-gui} (graphical user interface). Note that you need QT 4 or 5 for compiling the GUI version.
\item Run {\em build.sh clean} for removing all object and output files.
\item Run {\em build.sh dist} for building a distribution package.
\end{itemize}

You can change the behaviour and the target directories in the static configuration section, at the beggining of the build script.

\begin{lstlisting}
# Static configuration                                                          
build_pll=yes            # build PLL                                            
build_modules=yes        # build PLL modules library                            
build_gui=yes            # build modeltest-gui                                  
dir_base=${PWD}          # base directory                                       
prefix=${dir_base}/build # output directory for modeltest                       
qmake_bin=qmake          # qmake binary (for GUI)
\end{lstlisting}

\begin{tabular}{lll}
  {\bf property} & {\bf default} & {\bf description} \\
  \hline
  build\_pll & yes & Build the pll dependency \\
  build\_modules & yes & Build the pll modules dependency \\
  build\_gui & yes & Build the graphical user interface \\
  dir\_base & current directory & Base directory \\
  prefix & {\em dir\_base}/build & Target directory \\
  qmake\_bin & qmake & qmake binary \\
\end{tabular}
\vspace{1em}

Note: QT utility, {\em qmake} can be also an absolute/relative path, e.g., {\em qmake\_bin=/usr/local/bin/qmake}.
For example, I used a custom distribution for OSX and the configuration line looks like this:

\begin{lstlisting}
qmake_bin=/usr/local/Cellar/qt/5.9.1/bin/qmake
\end{lstlisting}

\subsection{Example run}

\begin{enumerate}
\item Execute the following command line:

\begin{lstlisting}
$ ./modeltest-ng -i example-data/dna/tiny.fas -h uigf -f ef
\end{lstlisting}

This will test all 88 models (gamma models with 4 rate categories), and then perform the model selection using Akaike (AIC) and Bayesian (BIC) criteria.

See Section~\ref{sec:arguments} for information about supported arguments.

\item This will generate the following output:

\begin{enumerate}

\item Header:

\begin{lstlisting}
                             _      _ _            _      _   _  _____ 
                            | |    | | |          | |    | \ | |/ ____|
         _ __ ___   ___   __| | ___| | |_ ___  ___| |_   |  \| | |  __ 
        | '_ ` _ \ / _ \ / _` |/ _ \ | __/ _ \/ __| __|  | . ` | | |_ |
        | | | | | | (_) | (_| |  __/ | ||  __/\__ \ |_   | |\  | |__| |
        |_| |_| |_|\___/ \__,_|\___|_|\__\___||___/\__|  |_| \_|\_____|
--------------------------------------------------------------------------------

modeltest 0.1.0
Copyright (C) 2017 Diego Darriba, David Posada, Alexandros Stamatakis
License GPLv3+: GNU GPL version 3 or later <http://gnu.org/licenses/gpl.html>.
This is free software: you are free to change and redistribute it.
There is NO WARRANTY, to the extent permitted by law.

Written by Diego Darriba.
--------------------------------------------------------------------------------
\end{lstlisting}

\item System and configuration details:

\begin{lstlisting}
Physical cores: 2
Logical cores:  4
Memory:         3.57GB
Extensions:     AVX

\end{lstlisting}

\item Execution options:

\begin{lstlisting}
--------------------------------------------------------------------------------
Input data:
  MSA:        example-data/dna/aP6.fas
  Tree:       Maximum parsimony
    file:     -
  #taxa:      6
  #sites:     631
  #patterns:  28

Output:
  Log:           test.log
  Starting tree: test.tree
  Results:       test.out

Selection options:
  # dna schemes:      11
  # dna models:       88
  include model parameters:
    Uniform:         true
    p-inv (+I):      true
    gamma (+G):      true
    both (+I+G):     true
    fixed freqs:     true
    estimated freqs: true
    #categories:     4
  asc bias:           none
  epsilon (opt):      0.01
  epsilon (par):      0.01

Additional options:
  verbosity:        very low
  threads:          1/2
  RNG seed:         12345
  subtree repeats:  enabled
--------------------------------------------------------------------------------
modeltest-ng was called as follows:
>> src/modeltest-ng -i example-data/dna/aP6.fas -h uifg -f fe -o test
\end{lstlisting}

\item Real time optimization results (progress):

\begin{lstlisting}
Partition 1/1

 ----ID---  ----MODEL---- ---Time--- -Elapsed--- -------LnL------- -Alpha- -P-inv-
    1/88   JC             0h:00:00   0h:00:00           -1115.1193       -       -
    2/88   JC+I           0h:00:00   0h:00:00           -1103.3444       -  0.9082
    3/88   JC+G           0h:00:00   0h:00:00           -1106.6136  0.0200       -
    4/88   JC+I+G         0h:00:00   0h:00:00           -1103.6235  1.1674  0.8542
    5/88   F81            0h:00:00   0h:00:00           -1065.0339       -       -
    6/88   F81+I          0h:00:00   0h:00:00           -1053.6319       -  0.9032
    7/88   F81+G          0h:00:00   0h:00:00           -1056.6126  0.0200       -
    8/88   F81+I+G        0h:00:00   0h:00:00           -1053.8953  1.1494  0.8460

...

   85/88   GTR            0h:00:00   0h:00:01           -1063.2358       -       -
   86/88   GTR+I          0h:00:00   0h:00:01           -1051.9056       -  0.9001
   87/88   GTR+G          0h:00:00   0h:00:01           -1054.7872  0.0200       -
   88/88   GTR+I+G        0h:00:00   0h:00:01           -1052.1689  1.1396  0.8417
 ----ID---  ----MODEL---- ---Time--- -Elapsed--- -------LnL------- -Alpha- -P-inv-

Computation of likelihood scores completed. It took 0h:00:01
\end{lstlisting}

\item Selected Information Criteria (best model and all models sorted according to each criterion):

\begin{lstlisting}
BIC       model              K            lnL          score          delta    weight
--------------------------------------------------------------------------------
       1  F81+I              4     -1053.6319      2191.0788         0.0000    0.8565
       2  HKY+I              5     -1053.1557      2196.5737         5.4949    0.0549
       3  F81+G              4     -1056.6126      2197.0401         5.9613    0.0435
       4  F81+I+G            5     -1053.8953      2198.0529         6.9741    0.0262
       5  TrN+I              6     -1052.6019      2201.9134        10.8346    0.0038
       6  TPM2uf+I           6     -1052.6600      2202.0296        10.9507    0.0036
       7  HKY+G              5     -1056.0996      2202.4615        11.3827    0.0029
       8  TPM3uf+I           6     -1052.9534      2202.6164        11.5376    0.0027
       9  TPM1uf+I           6     -1053.0742      2202.8579        11.7791    0.0024
      10  HKY+I+G            6     -1053.4340      2203.5777        12.4988    0.0017
--------------------------------------------------------------------------------
Best model according to BIC
---------------------------
Model:              F81+I
lnL:                -1053.6319
Frequencies:        0.4253 0.1506 0.2010 0.2232
Subst. Rates:       1.0000 1.0000 1.0000 1.0000 1.0000 1.0000
Inv. sites prop:    0.9032
Gamma shape:        -
Score:              2191.0788
Weight:             0.8565
---------------------------
Parameter importances
---------------------------
P.Inv:              0.9244
Gamma:              0.0471
Gamma-Inv:          0.0282
Frequencies:        1.0000
---------------------------
Model averaged estimates
---------------------------
P.Inv:              0.9031
Alpha:              0.0200
Alpha-P.Inv:        1.1502
P.Inv-Alpha:        0.8459
Frequencies:        0.4253 0.1506 0.2010 0.2232

Commands:
  > phyml  -i example-data/dna/aP6.fas -m 000000 -f m -v e -a 0 -c 1 -o tlr
  > raxmlHPC-SSE3 -s example-data/dna/aP6.fas -c 1 -m GTRCATIX --JC69 -n EXEC_NAME -p PARSIMONY_SEED
  > paup -s example-data/dna/aP6.fas
  > iqtree -s example-data/dna/aP6.fas -m F81+I
\end{lstlisting}

\item Consensus tree of the optimized phylogenies using the criterion weights (only for {\bf ML topologies}):

\begin{lstlisting}
  There are 2 different topologies
  Topologies written to output.topos

  topo_id   models_count   bic_support   aic_support   aicc_support
  -----------------------------------------------------------------
        1             37       0.95897       0.66064       0.66964
        2             51       0.04103       0.33936       0.33036

  extended majority-rule consensus: ((P4,(P6,P1)[1.00000])[0.95897],P5,(P2,P3)[1.00000]);
  strict consensus: ((P6,P1)[1.00000],P4,P5,(P2,P3)[1.00000]);
\end{lstlisting}

\end{enumerate}
\end{enumerate}
